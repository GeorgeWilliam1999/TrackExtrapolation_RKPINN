% =============================================================================
% Physics-Informed Neural Networks for Track Extrapolation in LHCb
% =============================================================================
\documentclass[11pt,a4paper]{article}

% Packages
\usepackage[utf8]{inputenc}
\usepackage[T1]{fontenc}
\usepackage{amsmath,amssymb,amsfonts}
\usepackage{graphicx}
\usepackage{booktabs}
\usepackage{hyperref}
\usepackage{cleveref}
\usepackage{siunitx}
\usepackage{subcaption}
\usepackage{xcolor}
\usepackage{algorithm}
\usepackage{algpseudocode}
\usepackage[margin=2.5cm]{geometry}

% Custom commands
\newcommand{\qop}{q/p}
\newcommand{\GeV}{\ensuremath{\,\text{GeV}}}
\newcommand{\MeV}{\ensuremath{\,\text{MeV}}}
\newcommand{\Tesla}{\ensuremath{\,\text{T}}}
\newcommand{\mm}{\ensuremath{\,\text{mm}}}
\newcommand{\mrad}{\ensuremath{\,\text{mrad}}}
\newcommand{\mus}{\ensuremath{\,\mu\text{s}}}
\newcommand{\ns}{\ensuremath{\,\text{ns}}}

% Title
\title{Physics-Informed Neural Networks for Fast and Accurate\\
Track Extrapolation in the LHCb Detector}

\author{G.~Scriven\textsuperscript{1}\\
\small \textsuperscript{1}Nikhef, Amsterdam, The Netherlands}

\date{\today}

% =============================================================================
\begin{document}
% =============================================================================

\maketitle

\begin{abstract}
Track extrapolation through the magnetic field is a computationally intensive 
component of high-energy physics reconstruction. Traditional methods like 
Runge-Kutta integration provide high accuracy but are limited by computational 
cost in high-rate environments. We present a systematic study of neural network 
approaches for track extrapolation in the LHCb detector, comparing standard 
Multi-Layer Perceptrons (MLP), Physics-Informed Neural Networks (PINN), and a 
novel Runge-Kutta-inspired Physics-Informed architecture (RK-PINN). Using 50 
million tracks generated with the true LHCb magnetic field map, we demonstrate 
that physics-informed approaches achieve [X]\% improvement in accuracy while 
maintaining inference speeds of [Y]$\mus$ per track. The RK-PINN architecture 
shows particular promise for low-momentum tracks where physics constraints are 
most important.
\end{abstract}

% Include sections
\input{sections/introduction}
\input{sections/data}
\input{sections/experimental_design}
% \input{sections/results}
% \input{sections/conclusion}

% Bibliography
\bibliographystyle{unsrt}
% \bibliography{references}

\end{document}
